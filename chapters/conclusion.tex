\chapter{\ifenglish Conclusions and Discussions\else บทสรุปและข้อเสนอแนะ\fi}

\section{\ifenglish Conclusions\else สรุปผล\fi}

% นศ. ควรสรุปถึงข้อจำกัดของระบบในด้านต่างๆ ที่ระบบมีในเนื้อหาส่วนนี้ด้วย
จากผลการดำเนินงานนี้สามารถบรรลุวัตถุประสงค์ของโครงงานในการช่วยให้ผู้ใช้งานสามารถหาที่นั่งได้สะดวกมากยิ่งขึ้นและสามารถวางแผนล่วงหน้าได้ อีกทั้งผลประเมินความพึงพอใจโดยรวมเฉลี่ยถือว่าอยู่ในระดับที่ดี
แต่ยังมีข้อจำกัดในด้านของการแสดงผลที่ไม่ทั่วถึงเนื่องจากโครงงานนี้ทดสอบแค่ 3 โซนของบริเวณชั้น 2 ในสำนักหอสมุดมหาวิทยาลัยเชียงใหม่

\section{\ifenglish Challenges\else ปัญหาที่พบและแนวทางการแก้ไข\fi}

% ในการทำโครงงานนี้ พบว่าเกิดปัญหาหลักๆ ดังนี้
% \subsection{ในช่วงแรก ESP32-CAM with OV2640 ไม่สามารถส่งภาพไปยัง AWS S3 ได้}
% เนื่องจากยังไม่คุ้นชินกับการใช้ฟังก์ชันต่างๆ ของ AWS Web Service ทำให้ไม่สามารถส่งภาพจาก ESP32-CAM with OV2640 ได้ จึงแก้ไขให้ ESP32-CAM with OV2640 ส่งภาพไปเก็บไว้ที่ Google Drive ก่อนแล้วค่อย upload ไปยัง AWS S3 ทำให้การทำงานล่าช้าในช่วงแรกและซ้ำซ้อน 
% \subsection{ESP32-CAM with OV2640 ไม่สามารถเชื่อมต่อ WiFi ของ JumboPlus IoT ได้}
\begin{enumerate}
    \item ไม่สามารถ upload code ไปที่ ESP32-CAM with OV2640 ได้ ทำให้เสียเวลากับส่วนนี้ค่อนข้างมากในการหาวิธีแก้ไข แนวทางการแก้ไขที่พบคือต้องทำการติดตั้ง driver CH340 ใหม่ทุกครั้งที่ขึ้น error
    \item ESP32-CAM with OV2640 ไม่สามารถเชื่อมต่อ WiFi ของ JumboPlus IoT ได้ จึงแก้ปัญหาโดยการแชร์ WiFi ของตนเองแทน แต่ในจุดที่มีคนใช้งานอินเทอร์เน็ตหนาแน่นบางที ESP32-CAM with OV2640 ก็จะไม่สามารถจับ WiFi ได้อยู่ดีซึ่งเป็นข้อจำกัดของกล้อง แนวทางการแก้ไขสามารถนำกล้องประเภทอื่นมาใช้งานแทนได้ เช่น กล้องวงจรปิด cctv
    \item ในช่วงแรก ESP32-CAM with OV2640 ไม่สามารถส่งภาพไปยัง AWS S3 ได้ เนื่องจากยังไม่คุ้นชินกับการใช้ฟังก์ชันต่างๆ ของ AWS Web Service 
    ทำให้ไม่สามารถส่งภาพจาก ESP32-CAM with OV2640 ได้ จึงแก้ไขให้ ESP32-CAM with OV2640 ส่งภาพไปเก็บไว้ที่ Google Drive ก่อนแล้วค่อย upload ไปยัง AWS S3 ทำให้การทำงานล่าช้าในช่วงแรกและซ้ำซ้อน แต่โครงงานขณะนี้สามารถเขียนโปรแกรมให้ส่งภาพไปที่ AWS S3 โดยตรงได้แล้ว
\end{enumerate}

\newpage
\section{\ifenglish%
Suggestions and further improvements
\else%
ข้อเสนอแนะและแนวทางการพัฒนาต่อ
\fi
}
% ข้อเสนอแนะเพื่อพัฒนาโครงงานนี้ต่อไป มีดังนี้
\subsection{ข้อเสนอแนะจากผู้ใช้งาน}
\begin{figure}[ht]
    \centering
    \includegraphics[scale=0.8]{images/student-s.png}
    \caption[ข้อเสนอแนะจากนักศึกษา]{ข้อเสนอแนะจากนักศึกษา}
    \label{fig:st-s}
% \end{figure}
% \begin{figure}[ht]
    \centering
    \includegraphics[scale=0.8]{images/cmul-s.png}
    \caption[ข้อเสนอแนะจากทางสำนักหอสมุด]{ข้อเสนอแนะจากทางสำนักหอสมุด}
    \label{fig:cmul-s}
\end{figure}

\subsection{ข้อเสนอแนะในการพัฒนาต่อ}
\begin{enumerate}
    \item ในส่วนของภาพที่ได้จาก ESP32-CAM with OV2640 สามารถนำไปทำ image processing ก่อนแล้วค่อยนำไปส่งต่อให้ AWS Rekcognition ทำการตรวจจับคน เพื่อเพิ่มประสิทธิภาพในการทำงานมากขึ้น เพราะบางทีเวลาติดตั้งกล้องอาจมีมุมกล้องไปซ้อนทับกับกล้องอีกตัวที่ติดตั้งใกล้ๆ กัน
    ควรจะ mask เฉพาะส่วนที่ต้องการไว้ AWS Rekcognition จะได้ตรวจสอบเฉพาะโซนที่ต้องการของกล้องนั้นๆ เท่านั้น ไม่ปนกับกล้องอื่น
    \item ใช้ motion detection ในการตรวจจับคนร่วมด้วยจะได้มีประสิทธิภาพมากยิ่งขึ้น
    \item ควรพัฒนาหน้าเว็บให้ทางสำนักหอสมุดสามารถจัดการได้ง่ายขึ้น เช่น ให้ staff login เข้าไปจัดการข้อมูลของกล้องแต่ละโซนมาแสดงผลได้ ไม่ต้องมาแก้ไขในโปรแกรมสามารถเลือกผ่านหน้าเว็บได้เลย
    \item เพิ่มฟังก์ชันการใช้งานในหน้าเว็บสำหรับผู้ใช้งานทั่วไปให้มีความหลากหลายมากยิ่งขึ้น เช่น มีการแจ้งเตือนเมื่อมีคนลุกจากที่นั่งแล้ว สามารถเลือกเช็คที่นั่งว่างที่ติดกันได้หากมาหลายคนว่ามีอยู่ที่โซนไหนบ้าง ระบบแจ้งเตือนว่ามีคนจองโต๊ะแล้วไม่อยู่ไปยังเจ้าหน้าที่สำนักหอสมุดให้มาดำเนินการต่อไป
    \item เปลี่ยนไอคอนในหน้าเว็บให้เข้าใจง่ายมากขึ้น เช่น ตรงที่ไม่มีกล้องแทนที่จะเป็นสีดำกับสัญลักษณ์ " - " ให้เป็นรูปกล้องที่โดนขีดทับว่าไม่มีแทน 
    \item พัฒนาหน้าเว็บให้ responsive กับหน้าจอสมาร์ทโฟน เนื่องจากตอนนี้หน้าเว็บสามารถแสดงไอคอนต่างๆ ได้ตรงตามที่ต้องการเฉพาะหน้าจอ desktop ของคอมพิวเตอร์และแท็บเล็ท หากดูในสมาร์ทโฟนต้องเปลี่ยนมุมมองเป็นหน้าจอ desktop แทน
    \item โครงงานนี้ได้มีการทดสอบการตรวจจับคนด้วย OpenCV with HOG descriptor, OpenCV with Detect common object library, OpenCV DNN with TensorFlow และ AWS Rekcognition จากการทดลองพบว่า AWS Rekcognition ตรวจจับคนได้ดีกว่าจึงเลือกนำมาใช้ในโครงงาน
    \item การใช้บริการของ AWS Web Service จะต้องมีการผูกบัตรเครดิต/เดบิตก่อนเริ่มใช้งาน จะไม่มีค่าใช้จ่ายเพิ่มเติมหากใช้งานไม่เกิน 5,000 รูปต่อเดือน
\end{enumerate}

\subsection{ค่าใช้จ่ายในการพัฒนาต่อ}
หากมีการติดตั้ง 60 จุดโดยให้ทำงานทุกๆ 1 นาที เป็นเวลา 7 ชั่วโมง เฉพาะวันจันทร์ถึงวันศุกร์เป็นเวลา 1 เดือน จะมีการประมวลรูปภาพ 756,000 รูปต่อเดือน ทำให้มีค่าใช้จ่ายดังตารางต่อไปนี้

 \begin{center}
    \begin{tabular}{ | m{5cm} | m{4cm}| } 
      \hline
      \textbf{Service} & \textbf{ค่าใช้จ่าย (บาท/เดือน)} \\ 
      \hline
      AWS Rekcognition & 35,301.68\\ 
      \hline
      AWS S3 & 0\\
      \hline
      AWS Lambda & 45.98\\
      \hline
      AWS IoT & 8.98\\
      \hline
      AWS DynamoDB & 0\\
      \hline
      AWS API Gateway & 34.12\\
      \hline
      \textbf{Total costs} & \textbf{35,390.76}\\
      \hline
    \end{tabular}
    \end{center}